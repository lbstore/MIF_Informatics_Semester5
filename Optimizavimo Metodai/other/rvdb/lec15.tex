%\PassOptionsToPackage{coloremph,colormath,colorhighlight,darkbackground,sans}{texpower}
\PassOptionsToPackage{coloremph,colormath,colorhighlight,whitebackground,sans}{texpower}
\input{__TPpble}
\input{hilite}
\hypersetup{pdftitle={ Linear Programming: Chapter 15 \\ 
Structural Optimization}}
\usepackage{amsmath}
\usepackage{times}
\usepackage{graphicx}
\delimitershortfall-1sp
\boldmath

\let\origbf=\bf
\renewcommand{\bf}{\origbf \color{blue}}

\newenvironment{mylist}{
		\begin{list}%
		{}{
		\parsep=0.2in
		\listparindent=0in
		\leftmargin=0.25in
		\rightmargin=0.25in
		}
	}{
		\end{list}
	}

\newenvironment{alignedt}{
                \setlength{\arraycolsep}{0.1em}
                \begin{array}[t]{rll}
                \displaystyle
        }{
                \end{array}
        }

\newenvironment{gequation}{
		\begin{equation*}
		\color{green}
	}{
		\end{equation*}
	}

\newenvironment{geqnarray}{
		\begin{eqnarray*}
		\color{green}
	}{
		\end{eqnarray*}
	}

\newcommand{\mysection}[1]{
		{\color{blue}
		\section{\rmfamily \LARGE #1}
		}
	}

\definecolor{orange}{rgb}{1,0.5,0}
\definecolor{subseccolor}{rgb}{0.2,0,0.7}

\newcommand{\mysubsection}[1]{
	\subsection*{\rmfamily \Large \color{subseccolor} #1}
}

\newtheorem{thm}{Theorem}
\newtheorem{cor}{Corollary}
\newtheorem{defn}{Definition}

\newcommand{\N}{{\mathcal N}}
\newcommand{\A}{{\mathcal A}}
\newlength{\hh}
\newcommand{\+}{\phantom{+}}

\newcommand{\Dx}{{\Delta x}}
\newcommand{\Dy}{{\Delta y}}
\newcommand{\Dw}{{\Delta w}}
\newcommand{\Dz}{{\Delta z}}

\begin{document}	   			
\begin{slide}
\title{\color{orange} Linear Programming: Chapter 16 \\ Interior-Point
Methods}
\author{\color{orange} Robert J. Vanderbei}
\maketitle
\vfill
        {Operations Research and Financial Engineering \\
	Princeton University \\
	Princeton, NJ 08544 \\
	%\protect\color{white}
	\href{http://www.princeton.edu/~rvdb}
	{\color{red} http://www.princeton.edu/$\sim$rvdb}
	}

\end{slide}

\begin{slide}
\mysection{Interior-Point Methods---The Breakthrough}
\hs{
\vfill

\begin{tabular}{rl}
\begin{minipage}{3in}
\begin{center}
\mbox{\includegraphics[width=3in]{../newsclip1.pdf}}
\end{center}
\end{minipage}
&
\begin{minipage}{3in}
\begin{center}
\mbox{\includegraphics[width=3in]{../newsclip2.pdf}}
\end{center}
\end{minipage}
\end{tabular}

%\vfill 
%}
%\end{slide}
%
%\begin{slide}
%\mysection{Time Magazine Joins In}
%\hs{
%\vfill


\vfill 
}
\end{slide}

\begin{slide}
\mysection{The Wall Street Journal Waits 'Till 1986}
\hs{
\vfill

\begin{tabular}{rl}
\begin{minipage}{3in}
\begin{center}
\mbox{\includegraphics[width=3in]{../newsclip3.pdf}}
\end{center}
\end{minipage}
&
\begin{minipage}{3in}
\begin{center}
\mbox{\includegraphics[width=3in]{../newsclip4.pdf}}
\end{center}
\end{minipage}
\end{tabular}


%\vfill 
%}
%\end{slide}
%
%\begin{slide}
%\mysection{Even Business Week Joins In}
%\hs{
%\vfill


\vfill 
}
\end{slide}

\begin{slide}
\mysection{AT\&T Patents the Algorithm, Announces KORBX}
\hs{
\vfill

\begin{tabular}{rl}
\begin{minipage}{3in}
\begin{center}
\mbox{\includegraphics[width=3in]{../newsclip5.pdf}}
\end{center}
\end{minipage}
&
\begin{minipage}{3in}
\begin{center}
\mbox{\includegraphics[width=3in]{../newsclip6.pdf}}
\end{center}
\end{minipage}
\end{tabular}


%\vfill 
%}
%\end{slide}
%
%\begin{slide}
%\mysection{AT\&T Announces the KORBX System}
%\hs{
%\vfill


\vfill 
}
\end{slide}

\begin{slide}
\mysection{What Makes LP Hard?}
\hs{
\vfill

\begin{tabular}{rl}
\begin{minipage}{3in}
{\em Primal}
\[
        \begin{array}{ll}
            \mbox{maximize }   & c^T x                          \\
            \mbox{subject to } & \begin{alignedt}
                                     Ax + w & = b \\
                                      x,w & \ge 0
                                 \end{alignedt}
        \end{array}
\]
\end{minipage}
&
\begin{minipage}{3in}
{\em Dual}
\[
        \begin{array}{ll}
            \mbox{minimize }   & b^T y                          \\
            \mbox{subject to } & \begin{alignedt}
                                     A^Ty - z & = c \\
                                       y,z  & \ge 0 
                                 \end{alignedt}
        \end{array}
\]
\end{minipage}
\end{tabular}

{\em Complementarity Conditions}

\begin{eqnarray*}
    x_j z_j & = & 0 \qquad j = 1,2,\ldots,n \\
    w_i y_i & = & 0 \qquad i = 1,2,\ldots,m \\
\end{eqnarray*}

\vfill 
}
\end{slide}

\begin{slide}
\mysection{Matrix Notation}
\hs{
\vfill

\begin{mylist}
    \item
	Can't write $x z = 0$.  The product $x z$ is undefined.
    \item
	Instead, introduce a new notation:
	\[
	    x =
	    \begin{bmatrix}
	        x_1 \\
	        x_2 \\
	        \vdots \\
	        x_n
	    \end{bmatrix}
	    \quad
	    \Longrightarrow
	    \quad
	    X =
	    \setlength{\arraycolsep}{0.4em}
	    \begin{bmatrix}
	        x_1 &     &        &     \\
	            & x_2 &        &     \\
	            &     & \ddots &     \\
	            &     &        & x_n
	    \end{bmatrix} 
	\]
    \item
	Then the complementarity conditions can be written as:
	\begin{eqnarray*}
	    X Z e & = & 0 \\
	    W Y e & = & 0 
	\end{eqnarray*}
\end{mylist}

\vfill 
}
\end{slide}

\begin{slide}
\mysection{Optimality Conditions}
\hs{
\vfill

\begin{eqnarray*}
     A  x + w & = & b \\
    A^T y - z & = & c \\
    Z X e & = & 0 \\
    W Y e & = & 0 \\
    w, x, y, z & \ge & 0 
\end{eqnarray*}

\begin{mylist}
    \item
	Ignore (temporarily) the nonnegativities.
    \item
	$2n+2m$ equations in $2n+2m$ unknowns.
    \item
	Solve'em.
    \item
	Hold on.  Not all equations are linear.
\end{mylist}

\begin{quote}
\bf
It is the nonlinearity of the complementarity conditions that makes LP
fundamentally harder than solving systems of equations.
\end{quote}

\vfill 
}
\end{slide}

\begin{slide}
\mysection{The Interior-Point Paradigm}
\hs{
\vfill

\begin{mylist}
    \item
	Since we're ignoring nonnegativities, it's best to
	replace complementarity with $\mu$-complementarity:
	\begin{eqnarray*}
	     A  x + w & = & b \\
	    A^T y - z & = & c \\
	    Z X e & = & \mu e \\
	    W Y e & = & \mu e 
	\end{eqnarray*}
    \item
	Start with an arbitrary (positive) initial guess: $x$, $y$, $w$, $z$.
    \item
	Introduce {\em step directions}: $\Dx$, $\Dy$, $\Dw$, $\Dz$.
    \item
	Write the above equations for $x+\Dx$, $y+\Dy$, $w+\Dw$, and $z+\Dz$:
	\begin{eqnarray*}
	     A  (x+\Dx) + (w+\Dw) & = & b \\
	    A^T (y+\Dy) - (z+\Dz) & = & c \\
	    (Z+\Delta Z) (X+\Delta X) e & = & \mu e \\
	    (W+\Delta W) (Y+\Delta Y) e & = & \mu e 
	\end{eqnarray*}
\end{mylist}

\vfill 
}
\end{slide}

\begin{slide}
\mysection{Paradigm Continued}
\hs{
\vfill

\begin{mylist}
    \item
	Rearrange with ``delta'' variables on left and drop nonlinear terms on
	left:
	\begin{eqnarray*}
	     A  \Dx + \Dw & = & b - Ax - w \\
	    A^T \Dy - \Dz & = & c - A^T y + z \\
	    Z \Dx + X \Dz & = & \mu e - ZXe \\
	    W \Dy + Y \Dw & = & \mu e - WYe
	\end{eqnarray*}
    \item
	This is a {\bf linear} system of $2m+2n$ equations in $2m+2n$
	unknowns.  
    \item
	Solve'em.
    \item
	Dampen the step lengths, if necessary, to maintain positivity.
    \item
	Step to a new point:
	\begin{eqnarray*}
	    x & \longleftarrow & x + \theta \Dx \\
	    y & \longleftarrow & y + \theta \Dy \\
	    w & \longleftarrow & w + \theta \Dw \\
	    z & \longleftarrow & z + \theta \Dz 
	\end{eqnarray*}
	($\theta$ is the scalar damping factor).
\end{mylist}

\vfill 
}
\end{slide}

\begin{slide}
\mysection{Paradigm Continued}
\hs{
\vfill

\begin{mylist}
    \item
	Pick a smaller value of $\mu$ for the next iteration.
    \item
	Repeat from beginning until current solution satisfies, within a
	tolerance, optimality conditions:
	\begin{description}
	    \item[primal feasibility] $b -  A  x - w = 0$.
	    \item[dual feasibility]   $c - A^T y + z = 0$.
	    \item[duality gap] $b^T y - c^T x = 0$.
	\end{description}
\end{mylist}

{\bf Theorem.} 
\begin{itemize}
    \item
	Primal infeasibility gets smaller by a factor of $1 - \theta$ at 
	every iteration.
    \item
	Dual infeasibility gets smaller by a factor of $1 - \theta$ at 
	every iteration.
    \item
	If primal and dual are feasible, then duality gap decreases by a factor
	of $1 - \theta$ at every iteration (if $\mu = 0$, slightly slower
	convergence if $\mu > 0$).
\end{itemize}

\vfill 
}
\end{slide}

\begin{slide}
\mysection{\sc loqo}
\vfill

\begin{mylist}
    \item
	Hard/impossible to ``do'' an interior-point method by hand.
    \item
	Yet, easy to program on a computer (solving large systems of equations
	is routine).
    \item
	LOQO implements an interior-point method.
    \item
	Setting {\tt option loqo\_options 'verbose=2' }
        in AMPL produces the following ``typical'' output:
\end{mylist}

\vfill 
\end{slide}

\begin{slide}
\mysection{{\sc loqo} Output}
\vfill
{
\scriptsize
\begin{verbatim}
variables: non-neg   1350,  free        0,  bdd          0,  total     1350
constraints: eq       146,  ineq        0,  ranged       0,  total      146
nonzeros:    A       5288,  Q           0
nonzeros:    L       7953,  arith_ops             101444
---------------------------------------------------------------------------
      |         Primal            |           Dual            | Sig        
Iter  |  Obj Value      Infeas    |   Obj Value      Infeas   | Fig  Status
- - - - - - - - - - - - - - - - - - - - - - - - - - - - - - - - - - - - - -
   1  -7.8000000e+03   1.55e+03     5.5076028e-01   4.02e+01               
   2   2.6725737e+05   7.84e+01     1.0917132e+00   1.65e+00               
   3   1.1880365e+05   3.92e+00     4.5697310e-01   2.02e-13             DF
   4   6.7391043e+03   2.22e-01     7.2846138e-01   1.94e-13             DF
   5   9.5202841e+02   3.12e-02     5.4810461e+00   1.13e-14             DF
   6   2.1095320e+02   6.03e-03     2.7582307e+01   4.15e-15             DF
   7   8.5669013e+01   1.36e-03     4.2343105e+01   2.48e-15             DF
   8   5.8494756e+01   3.42e-04     4.6750024e+01   2.73e-15     1       DF
   9   5.1228667e+01   8.85e-05     4.7875326e+01   2.59e-15     1       DF
  10   4.9466277e+01   2.55e-05     4.8617380e+01   2.86e-15     2       DF
  11   4.8792989e+01   1.45e-06     4.8736603e+01   2.71e-15     3    PF DF
  12   4.8752154e+01   7.26e-08     4.8749328e+01   3.36e-15     4    PF DF
  13   4.8750108e+01   3.63e-09     4.8749966e+01   3.61e-15     6    PF DF
  14   4.8750005e+01   1.81e-10     4.8749998e+01   2.91e-15     7    PF DF
  15   4.8750000e+01   9.07e-12     4.8750000e+01   3.21e-15     8    PF DF
----------------------
OPTIMAL SOLUTION FOUND
\end{verbatim}
}

\vfill 
\end{slide}

\begin{slide}
\mysection{A Generalizable Framework}
\hs{
\vfill

\begin{tabular}[t]{ll}
\begin{minipage}[t]{3in}
	Start with an optimization problem---in this case LP:
\end{minipage}
&
\begin{minipage}[t]{3in}
	\[
		\begin{array}{ll}
		    \mbox{maximize }   & c^T x                          \\
		    \mbox{subject to } & \begin{alignedt}
					     Ax & \le b \\
					      x & \ge 0
					 \end{alignedt}
		\end{array}
	\]
\end{minipage}
\end{tabular}

\begin{tabular}[t]{ll}
\begin{minipage}[t]{3in}
	Use slack variables to make all inequality constraints into 
	nonnegativities:
\end{minipage}
&
\begin{minipage}[t]{3in}
	\[
		\begin{array}{ll}
		    \mbox{maximize }   & c^T x                          \\
		    \mbox{subject to } & \begin{alignedt}
					     Ax + w & = b \\
					      x,w & \ge 0
					 \end{alignedt}
		\end{array}
	\]
\end{minipage}
\end{tabular}

\begin{mylist}
    \item
	Replace nonnegativity constraints with {\em logarithmic barrier terms}
	in the objective:

	\[
		\begin{array}{ll}
		    \mbox{maximize }   & c^T x 
					 + \mu \sum_j \log x_j 
					 + \mu \sum_i \log w_i \\
		    \mbox{subject to } & \begin{alignedt}
					     Ax + w & = b 
					 \end{alignedt}
		\end{array}
	\]
    \item
	Incorporate the equality constraints into the objective using {\em
	Lagrange multipliers}:

	\[
	    L(x,w,y) = 
	    c^T x + \mu \sum_j \log x_j + \mu \sum_i \log w_i 
	     + y^T(b - Ax - w) 
	\]
    \item
	Set derivatives to zero:
	\begin{eqnarray*}
	    c + \mu X^{-1} e - A^T y & = & 0 \qquad \mbox{(deriv wrt $x$)} \\
	        \mu W^{-1} e -     y & = & 0 \qquad \mbox{(deriv wrt $w$)} \\
	            b - A x - w      & = & 0 \qquad \mbox{(deriv wrt $y$)} 
	\end{eqnarray*}
    \item
	Introduce {\em dual complementary variables}:
	\[
	    z = \mu X^{-1} e
	\]
    \item
	Rewrite system:
	\begin{eqnarray*}
	           c + z - A^T y & = & 0 \\
			  X Z e  & = & \mu e \\
			  W Y e  & = & \mu e \\
	            b - A x - w  & = & 0 
	\end{eqnarray*}
\end{mylist}

\vfill 
}
\end{slide}

\begin{slide}
\mysection{Logarithmic Barrier Functions}
\hs{
\vfill

Plots of $\mu \log x$ for various values of $\mu$:

\begin{center}
\text{\includegraphics[width=3in]{../mu_log.pdf}}
\end{center}

\vfill 
}
\end{slide}

\begin{slide}
\mysection{Lagrange Multipliers}
\hs{
\vfill

\[
        \begin{array}{ll}
            \text{maximize }   & f(x) \\
            \text{subject to } & g(x) = 0 
        \end{array}
\]

\begin{center}
\text{\includegraphics[width=4in]{../lagrange1.pdf}}
\end{center}

\[
        \begin{array}{ll}
            \text{maximize }   & f(x) \\
            \text{subject to } & \begin{alignedt}
                                     g_1(x) & = 0 \\
                                     g_2(x) & = 0 
                                 \end{alignedt}
        \end{array}
\]

\begin{center}
\text{\includegraphics[width=4in]{../lagrange2.pdf}}
\end{center}

\vfill 
}
\end{slide}


\end{document}
